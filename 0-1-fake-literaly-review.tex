
\chapter{Аналитический раздел}

В большинстве случаев, для распознавания рукописного текста используются нейронные сети. Основные преимущества нейронных сетей в решении данной задачи состоят в способности обучаться самостоятельно и автоматически на основе выборок, быть продуктивными на зашумленных данных, возможностью параллельной реализации и эффективными инструментами для обработки больших баз данных.

В статье Князева \cite{fl1} применение нейронных сетей в распознавании рукописного текста предлагается использовать нейронную сеть, которую можно будет обучать с учителем и самостоятельно калибровать уровень ее надежности.

В публикации “Идентификация подписи с помощью радикальных функций” \cite{fl2} описывается исследование идентификации on-line подписи с помощью радиальных функций (определенные в евклидовом пространстве Rn, значение которых в каждой точке зависит только от расстояния между этой точкой и началом координат) и вейвлетов (математические функции, позволяющая анализировать различные частотные компоненты данных). Подробно описывается представление подписи в виде функции, применение для ее описания радиальных функций и вейвлет преобразований, также приведено сравнение полученных функций подписей одного автора, разных людей с целью идентификации подписей.

В статье Исрафилова Х. С. \cite{fl3} рассматривается задача распознавания слитного рукописного текста, а также обсуждаются различные подходы к решению данной задачи, отмечаются достоинства и недостатки рассматриваемых подходов. Также предлагается комбинированный подход к распознаванию слитно написанного рукописного слова, включающего в себя процедуру разбиения, основанную на анализе структуры слова, и процедуру распознавания, основанную на использовании нейронной сети.

В публикации \cite{fl4} рассматривается задача распознавания текста на китайском языке на основе ключей с использованием нейронных сетей. В ней рассмотрены основные этапы процесса распознавания рукописного текста, также приведены результаты исследования. Разработанный в данном исследовании алгоритм производит оффлайн распознавание рукописного текста на основе разбиения иероглифа на ключи и распознавании ключей с помощью неокогнитрона.

Описываемое приложение позволит значительно упростить работу переводчиков, а также поможет при изучении китайского языка. Минус предложенного метода заключается в сложности реализации системы и отсутствия возможности накопления знаний.

Исходя из предложенных ниже методов, можно сделать вывод о наличии заметных ограничений практической применимости известных методов для качественного адаптивного распознавания рукописного текста.
