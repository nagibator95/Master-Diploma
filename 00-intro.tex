
Информатикс -- информационный ресурс (далее -- сайт), расположенный в сети Интернет 
и доступный по доменным адресам \texttt{informatics.msk.ru} и \texttt{informatics.mccme.ru}.

Сайт представляет собой сборник задач по программированию различной сложности, от уровня начинающих и до задач уровня международных олимпиад. 
Задачи можно решать -- прогонять на тестах программный код, который протестируется системой. 
После прохождения тестов пользователю сайта выдается результат тестирования в виде количества пройденных тестов, 
статусе успешности или неуспешностирешения задачи, сообщение копилятора или интерпретатора.
Помимо этого, сайт представляет фунционал создания курсов и уроков на основе существующих задач.
Также у сайта есть функционал управления обучением (англ. Learnging Management System -- далее по тексту LMS).

Информатикс был создан в 2007 году и имеет более чем 10-летнюю историю.
Сайт имеет огромную базу задач, более 6 тысяч, 
что делает его одной из крупнейших русскоязычных баз задач по программированию, в которой накоплены материалы проводившихся за многие годы олимпиадных мероприятий. 
Стоимость базы задач Информатикс быда оценена экспетами примерно в 60 миллионов рублей. 
В то же время, на большую часть задач распространиются различные копирайт-условия, разрешающие их использование только на ресурсе Информатикс.

Помимо этого, важной частью Информатикс является контент курсов и уроков, 
который также был накоплен за годы существования ресурса.

Также Информатикс используется для различных сборов, например, Летняя Олимпиадная Школа (ЛОШ), 
и для проведения различных отборов, например, отбор в Образовательный центр "Сириус" по дисциплине программирование\cite{inf_tinkoff}.

К сожалению, с течением времени функционал сайта становился всё менее стабильным, 
из-за глубинных архитектурных проблем им становилось всё сложнее пользоваться\cite{inf_not_working}.

Стоит также отметить, что Информатикс принадлежит организации ГБОУ ЦПМ, что расшифровывается как Центр Педогагической Подготовки,
и бюджет организации имеет серьёзное органичение на ресурсы, выделяемые на аренду и покупку аппаратных мощностей.

Так же важным требованием к системе можно считать необходимость практически полной доступности сайта (Доступность сайта должна стремиться к 100\%), 
а все технические работы проходить без простоев основной функциональности сайта.

Помимо вышесказанного, существует потенциал Информатикс для использования его как базы задач и систему тестирования в сторонних проектах и системах.