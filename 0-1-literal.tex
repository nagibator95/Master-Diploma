\chapter{АНАЛИТИЧЕСКИЙ ОБЗОР ЛИТЕРАТУРЫ}

Информатикс -- важная часть экосистемы Российского олимпиадного программирования.
Ресурс используется при подготовке учеников к различным олимпиадам,
а так а так же для проведения уроков информатики в Российских школах.

Информатикс был создан в 2007 году и имеет более чем 10-летнюю историю.
Сайт имеет огромную базу задач, более 6 тысяч, 
что делает его одной из крупнейших русскоязычных баз задач по программированию, в которой накоплены материалы проводившихся за многие годы олимпиадных мероприятий. 
Стоимость базы задач Информатикс быда оценена экспетами примерно в 60 миллионов рублей. 

В то же время, на большую часть задач распространиются различные копирайт-условия, разрешающие их использование только на ресурсе Информатикс.

Помимо этого, важной частью Информатикс является контент курсов и уроков, 
который также был накоплен за годы существования ресурса.

Также Информатикс используется для различных сборов, например, Летняя Олимпиадная Школа (ЛОШ), 
и для проведения различных отборов, например, отбор в Образовательный центр "Сириус" по дисциплине программирование\cite{inf_tinkoff}.

Информатикс также используется в целях самоподготовки.
Ученики могут найти контест, использовавшийся в каком либо соревновании и отборе и самостоятельно решить его.
Также они могут найти и отдельную задачу из базы Информатикс и решить её отдельно.

Отдельной ценностью является база посылок, где хранится более 15 миллионов протестированных исходных кодов решения задач -- посылок --, их результатов и протоколов тестирования. 

Стоит также отметить, что Информатикс принадлежит организации ГБОУ ЦПМ, что расшифровывается как Центр Педогагической Подготовки,
и бюджет организации имеет серьёзное органичение на ресурсы, выделяемые на аренду и покупку аппаратных мощностей.

Так же важным требованием к системе можно считать необходимость практически полной доступности сайта (Доступность сайта должна стремиться к 100\%), 
а все технические работы проходить без простоев основной функциональности сайта.

Помимо вышесказанного, существует потенциал Информатикс для использования его как базы задач и систему тестирования в сторонних проектах и системах.

\section{Существующие аналоги}

Аналогами Информатикс с точки зрения проведения контестов
-- соревнований по спортивному программированию -- 
можно считать ресурсы Codeforces, Яндекс.Контест и другие.

Однако ресурс Яндекс.Контест не предоставляет возможностей для самостоятельного прохождения контестов -- после окончания регистрации на соревнование порешать контест уже не получится.

Также Яндекс.Контест не предоставляет возможности решать задачи не в рамках контеста.
Помимо этого, ресурс был создан компанией Яндекс в своих утилитарных целях,
и политика ресурса не даёт возможности свободно создавать свои задачи
и проводить свои контесты, как это можно делать на Информатикс.

Codeforces -- российский сайт олимпиадного программирования, известный и за рубежом\cite{codeforces_countries}.
Codeforces в отличие от Яндекс.Контест предоставляет возможности по самостоятельной подготовке в рамках решения отдельных задач, однако нельзя решить целый контест.

Также стоит отметить, что Codeforces -- частная компания, 
и для создания и проведения контеста может потребоваться заплатить.
В то же время, Информатикс совершенно бесплатен для своих пользователей.

Аналогами Информатикс с точки зрения базы олимпиадных задач 
можно считать ресурсы Codeforces и Tumus.

