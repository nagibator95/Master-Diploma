\chapter{\MakeUppercase{Заключение}}

В рамках данной работы был проведён аналитический обзор литературы.
В работе было представлено описание текущих архитектурных проблем Информатикс и
была проведена работа по постановке архитектурных проблем.

Была описана существующая архитектура. 
В рамках описания существующей архитектуры были описаны основные её части, такие, как Moodle, py-ручки и Ejudge.
Была совершена содержательная постановка задачи. 
В работе были рассмотрены существующие архитектурные проблемы Информатикс.

Затем были описаны общие принципы решения, 
описаны архитектурные реализации как отдельных систем, так и новой общей архитектуры Информатикс.

Были обоснованы применённые архитектурные изменения, 
было разработано две новые системы, Rmatics и ejudge-listener,
были описаны общие принципы функционирования этих систем.

Было обосновано изменение одного из основных хранилищ данных, Файловой системы, на MongoDB.

Также было описано взаимодействие между Rmatics и ejudge-listener 
и другими частями Информатикс,
такими, как СУБД MySQL, СУБД MongoDB, СУБД Redis и py-ручками.

Был разработан пошаговый план по переходу на новую архитектуру,
обеспечивающий максимальную доступность сервиса, 
а также обеспечивающий уверенность в принятных архитектурных решениях.

План включал в себя как описание разворачивания основных систем, 
так и шаги по последовательному встраиванию новых систем в существующую систему.
Этот план был исполнен, а архитектура Информатикс была обновлена.
